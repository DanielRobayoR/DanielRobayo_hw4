\documentclass[12pt]{article} 
\usepackage{graphicx}
\author{Daniel Robayo Reyes}
\date{11 de Noviembre de 2017}
\title{Resultados Solucion Ecuacion de Onda}
\begin{document}
\maketitle
\section{Una Dimension:}


\includegraphics[scale=0.70]{figuras1d_caso1.png}

Aqui se puede observar el comportamiento de la cuerda para diferentes tiempos. La cuerda tiene los extremos fijos.

\includegraphics[scale=0.70]{figuras1d_caso2.png}

Aqui se puede observar el comportamiento de la cuerda para diferentes tiempos. La cuerda tiene en un extremo una perturbacion de tipo sinusoidal. El extremo que tiene la perturbacion es el extremo izquierdo.

\section{Dos Dimensiones:}

\subsection{Condicion inicial}

\includegraphics[scale=0.70]{figuras2d_Condi_Inicial.png}

Aqui se puede observar el comportamiento de la membrana del tambor en el tiempo inicial (t=0). La membrana tiene todos sus extremos fijos.

\subsection{Tiempo 1}
\includegraphics[scale=0.70]{figuras_2d_Temp1.png}

Aqui se puede observar el comportamiento de la membrana del tambor en un tiempo posterior. La membrana tiene todos sus extremos fijos.

\subsection{Tiempo 2}
\includegraphics[scale=0.70]{figuras_2d_Temp2.png}

Aqui se puede observar el comportamiento de la membrana del tambor en un tiempo posterior. La membrana tiene todos sus extremos fijos.


\subsection{Tiempo 3}
\includegraphics[scale=0.70]{figuras_2d_Temp3.png}

Aqui se puede observar el comportamiento de la membrana del tambor en un tiempo posterior. La membrana tiene todos sus extremos fijos.



\end{document}
